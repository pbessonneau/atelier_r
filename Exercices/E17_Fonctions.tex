\documentclass{article}\usepackage[]{graphicx}\usepackage[]{color}
%% maxwidth is the original width if it is less than linewidth
%% otherwise use linewidth (to make sure the graphics do not exceed the margin)
\makeatletter
\def\maxwidth{ %
  \ifdim\Gin@nat@width>\linewidth
    \linewidth
  \else
    \Gin@nat@width
  \fi
}
\makeatother

\definecolor{fgcolor}{rgb}{0.345, 0.345, 0.345}
\newcommand{\hlnum}[1]{\textcolor[rgb]{0.686,0.059,0.569}{#1}}%
\newcommand{\hlstr}[1]{\textcolor[rgb]{0.192,0.494,0.8}{#1}}%
\newcommand{\hlcom}[1]{\textcolor[rgb]{0.678,0.584,0.686}{\textit{#1}}}%
\newcommand{\hlopt}[1]{\textcolor[rgb]{0,0,0}{#1}}%
\newcommand{\hlstd}[1]{\textcolor[rgb]{0.345,0.345,0.345}{#1}}%
\newcommand{\hlkwa}[1]{\textcolor[rgb]{0.161,0.373,0.58}{\textbf{#1}}}%
\newcommand{\hlkwb}[1]{\textcolor[rgb]{0.69,0.353,0.396}{#1}}%
\newcommand{\hlkwc}[1]{\textcolor[rgb]{0.333,0.667,0.333}{#1}}%
\newcommand{\hlkwd}[1]{\textcolor[rgb]{0.737,0.353,0.396}{\textbf{#1}}}%

\usepackage{framed}
\makeatletter
\newenvironment{kframe}{%
 \def\at@end@of@kframe{}%
 \ifinner\ifhmode%
  \def\at@end@of@kframe{\end{minipage}}%
  \begin{minipage}{\columnwidth}%
 \fi\fi%
 \def\FrameCommand##1{\hskip\@totalleftmargin \hskip-\fboxsep
 \colorbox{shadecolor}{##1}\hskip-\fboxsep
     % There is no \\@totalrightmargin, so:
     \hskip-\linewidth \hskip-\@totalleftmargin \hskip\columnwidth}%
 \MakeFramed {\advance\hsize-\width
   \@totalleftmargin\z@ \linewidth\hsize
   \@setminipage}}%
 {\par\unskip\endMakeFramed%
 \at@end@of@kframe}
\makeatother

\definecolor{shadecolor}{rgb}{.97, .97, .97}
\definecolor{messagecolor}{rgb}{0, 0, 0}
\definecolor{warningcolor}{rgb}{1, 0, 1}
\definecolor{errorcolor}{rgb}{1, 0, 0}
\newenvironment{knitrout}{}{} % an empty environment to be redefined in TeX

\usepackage{alltt}
% \usetheme[compress]{Singapore}
% \useoutertheme{miniframes}

% \documentclass{beamer}
%\usetheme{Warsaw}

% Pour les documents en francais...
	\usepackage[latin1]{inputenc}
	\usepackage[french]{babel}
	\usepackage[french]{varioref}

% Math?matiques
	\usepackage{amsmath}

% Caracteres speciaux suppl?mentaires
	\usepackage{latexsym,amsfonts}

% A documenter
	\usepackage{moreverb}

% Macros pour les paquets
	\usepackage{array}  			% N?cessaires pour les tableaux de la macro Excel.

% Outil suppl?mentaire pour les tableaux
	\usepackage{multirow}
	\usepackage{booktabs}
	\usepackage{xcolor} % alternating row colors in table, incompatible avec certains modules
	\usepackage{longtable}
	\usepackage{colortbl}

% Pour ins?rer des graphiques
	\usepackage{graphicx} 			% Graphique simples
	\usepackage{subfigure}			% Graphiques multiples

% Pour ins?rer des couleurs
	\usepackage{color}

% Rotation des objets et des pages
%	\usepackage{rotating}
%	\usepackage{lscape}

% Pour insrer du code source, LaTeX ou SAS par exemple.
	\usepackage{verbatim}
        \usepackage{moreverb}
	\usepackage{listings}
	\usepackage{fancyvrb}

%	\lstset{language=SAS,numbers=left}		% Par dfaut le listing est en SAS

% Pour ins?rer des hyperliens
  \usepackage{hyperref}

% American Psychological Association (for bibliographic references).
	\usepackage{apacite}

% Pour l'utilisation des macros
	\usepackage{xspace}

% Pour l'utilisation de notes en fin de document.
%	\usepackage{endnotes}

% Array
%	\usepackage{multirow}
%	\usepackage{booktabs}

% Rotation
%	\usepackage{rotating}

% En t?tes et pieds de pages
%	\usepackage{fancyhdr}
%	\usepackage{lastpage}


% Page layout

% By LaTeX commands
%\setlength{\oddsidemargin}{0cm}
%\setlength{\textwidth}{16cm}
%\setlength{\textheight}{24cm}
%\setlength{\topmargin}{-1cm}
%\setlength{\marginparsep}{0.2cm}

% fancyheader parameters
%\pagestyle{fancy}

%\fancyfoot[L]{{\small Formation \LaTeX, DEPP}}
%\fancyfoot[c]{}
%\fancyfoot[R]{{\small \thepage/\pageref{LastPage}}}

%\fancyhead[L]{}
%\fancyhead[c]{}
%\fancyhead[R]{}

% Pour ins?rer des dessins de Linux
\newcommand{\LinuxA}{\includegraphics[height=0.5cm]{Graphiques/linux.png}}
\newcommand{\LinuxB}{\includegraphics[height=0.5cm]{Graphiques/linux.png}\xspace}

% Macro pour les petits dessins pour les diff?rents OS.
\newcommand{\Windows}{\emph{Windows}\xspace}
\newcommand{\Mac}{\emph{Mac OS X}\xspace}
\newcommand{\Linux}{\emph{Linux}\xspace}
\newcommand{\MikTeX}{MiK\tex\xspace}
\newcommand{\latex}{\LaTeX\xspace}


\newcommand{\df}{\emph{data.frame}\xspace}
\newcommand{\liste}{\emph{list}\xspace}
\newcommand{\cad}{c'est-�-dire\xspace}

% Titre
\title{Introduction � R\\Exercices sur les fonctions}
\author{Pascal Bessonneau}
%\institute{DEPP}
\date{06/2015}
%\subtitle{RStudio}


\newcommand{\hreff}[2]{\underline{\href{#1}{#2}\xspace}}



\IfFileExists{upquote.sty}{\usepackage{upquote}}{}
\begin{document}

\section{Exercices sur les fonctions}

  \begin{enumerate}
    \item Ecrire une fonction qui fait un histogramme avec des valeurs pour les axes
    en fran�ais et des barres de couleur rouge
    \item Ecrire une fonction qui remplace les points des chiffres num�riques par 
    des virgules apr�s avoir arrondi � la d�cimale voulue.
    \item Appliquer cette derni�re fonction sur une matrice 3x3 de nombres tir�s
    dans la loi normale de moyenne 100 et d'�cart-type 50
    \item Ecrire une fonction "summarize" qui renvoie dans l'ordre, les quantiles 
    (fonction \emph{quantile}), la moyenne et l'�cart-type pour une variable
    \item Ecrire le code permettant le plus efficacement possible de remplacer
    dans le dataset \emph{setosa} par "typeA", \emph{versicolor} par "typeB" et 
    \emph{virginica} par "typeC". 
  \end{enumerate}
  


\clearpage

\section{Petit exercice 1}


\begin{knitrout}\footnotesize
\definecolor{shadecolor}{rgb}{0.969, 0.969, 0.969}\color{fgcolor}\begin{kframe}
\begin{alltt}
\hlstd{> }\hlstd{my.hist} \hlkwb{<-} \hlkwa{function}\hlstd{(}\hlkwc{x}\hlstd{,}\hlkwc{col}\hlstd{=}\hlstr{"red"}\hlstd{,}\hlkwc{ylab}\hlstd{=}\hlstr{"Effectifs"}\hlstd{,}\hlkwc{...}\hlstd{) \{}
\hlstd{+ }  \hlkwd{hist}\hlstd{(x,...,}\hlkwc{col}\hlstd{=}\hlstr{"red"}\hlstd{,}\hlkwc{ylab}\hlstd{=lab)}
\hlstd{+ }\hlstd{\}}
\hlstd{> }
\hlstd{> }\hlstd{my.hist} \hlkwb{<-} \hlkwa{function}\hlstd{(}\hlkwc{x}\hlstd{,}\hlkwc{col}\hlstd{=}\hlstr{"red"}\hlstd{,}\hlkwc{ylab}\hlstd{=}\hlkwa{NULL}\hlstd{,}\hlkwc{freq}\hlstd{=T,}\hlkwc{...}\hlstd{) \{}
\hlstd{+ }  \hlkwa{if} \hlstd{(freq) ylab} \hlkwb{<-} \hlstr{"Effectif"}
\hlstd{+ }  \hlkwa{else} \hlstd{ylab} \hlkwb{<-} \hlstr{"Densit�"}
\hlstd{+ }  \hlkwd{hist}\hlstd{(x,...,}\hlkwc{freq}\hlstd{=freq,}\hlkwc{col}\hlstd{=}\hlstr{"red"}\hlstd{,}\hlkwc{ylab}\hlstd{=ylab)}
\hlstd{+ }\hlstd{\}}
\end{alltt}
\end{kframe}
\end{knitrout}



\section{Petit exercice 2}

\begin{knitrout}\footnotesize
\definecolor{shadecolor}{rgb}{0.969, 0.969, 0.969}\color{fgcolor}\begin{kframe}
\begin{alltt}
\hlstd{> }\hlstd{francais} \hlkwb{<-} \hlkwa{function}\hlstd{(}\hlkwc{x}\hlstd{,}\hlkwc{digits}\hlstd{=}\hlnum{2}\hlstd{) \{}
\hlstd{+ }  \hlkwd{gsub}\hlstd{(}\hlstr{"."}\hlstd{,}\hlstr{","}\hlstd{,}\hlkwd{round}\hlstd{(x,digits),}\hlkwc{fixed}\hlstd{=}\hlnum{TRUE}\hlstd{)}
\hlstd{+ }\hlstd{\}}
\end{alltt}
\end{kframe}
\end{knitrout}

\begin{knitrout}\footnotesize
\definecolor{shadecolor}{rgb}{0.969, 0.969, 0.969}\color{fgcolor}\begin{kframe}
\begin{alltt}
\hlstd{> }\hlkwd{francais}\hlstd{(}\hlkwd{rnorm}\hlstd{(}\hlnum{5}\hlstd{,}\hlnum{100}\hlstd{,}\hlnum{50}\hlstd{))}
\end{alltt}
\end{kframe}
\end{knitrout}

\section{Petit exercice 3}

\begin{knitrout}\footnotesize
\definecolor{shadecolor}{rgb}{0.969, 0.969, 0.969}\color{fgcolor}\begin{kframe}
\begin{alltt}
\hlstd{> }\hlstd{a}\hlkwb{=}\hlkwd{matrix}\hlstd{(}\hlkwd{rnorm}\hlstd{(}\hlnum{3}\hlopt{^}\hlnum{2}\hlstd{,}\hlnum{100}\hlstd{,}\hlnum{50}\hlstd{),}\hlkwc{ncol}\hlstd{=}\hlnum{3}\hlstd{,}\hlkwc{nrow}\hlstd{=}\hlnum{3}\hlstd{)}
\hlstd{> }\hlkwd{apply}\hlstd{(a,}\hlnum{1}\hlopt{:}\hlnum{2}\hlstd{,francais)}
\end{alltt}
\end{kframe}
\end{knitrout}



\section{Petit exercice 4}


\begin{knitrout}\footnotesize
\definecolor{shadecolor}{rgb}{0.969, 0.969, 0.969}\color{fgcolor}\begin{kframe}
\begin{alltt}
\hlstd{> }\hlstd{summarize} \hlkwb{<-} \hlkwa{function}\hlstd{(}\hlkwc{x}\hlstd{,}\hlkwc{na.rm}\hlstd{=T...) \{}
\hlstd{+ }  \hlkwd{return}\hlstd{(}\hlkwd{c}\hlstd{(}
\hlstd{+ }    \hlkwd{quantile}\hlstd{(x,}\hlkwc{na.rm}\hlstd{=T),}
\hlstd{+ }    \hlkwc{Moy}\hlstd{=}\hlkwd{mean}\hlstd{(x,}\hlkwc{na.rm}\hlstd{=T),}
\hlstd{+ }    \hlkwc{EC}\hlstd{=}\hlkwd{sd}\hlstd{(x,}\hlkwc{na.rm}\hlstd{=T)}
\hlstd{+ }  \hlstd{))}
\hlstd{+ }\hlstd{\}}
\hlstd{> }\hlkwd{t}\hlstd{(}\hlkwd{sapply}\hlstd{(iris[,}\hlnum{1}\hlopt{:}\hlnum{4}\hlstd{],summarize))}
\end{alltt}
\end{kframe}
\end{knitrout}


\section{Petit exercice 4}

\begin{knitrout}\footnotesize
\definecolor{shadecolor}{rgb}{0.969, 0.969, 0.969}\color{fgcolor}\begin{kframe}
\begin{alltt}
\hlstd{> }\hlkwd{summarize}\hlstd{(iris}\hlopt{$}\hlstd{Sepal.Width)}
\end{alltt}
\end{kframe}
\end{knitrout}


\section{Petit exercice 5}

\begin{knitrout}\footnotesize
\definecolor{shadecolor}{rgb}{0.969, 0.969, 0.969}\color{fgcolor}\begin{kframe}
\begin{alltt}
\hlstd{> }\hlstd{a} \hlkwb{=} \hlkwd{c}\hlstd{(} \hlkwc{setosa} \hlstd{=} \hlstr{"typeA"}\hlstd{,} \hlkwc{versicolor} \hlstd{=} \hlstr{"typeB"}\hlstd{,}
\hlstd{+ }       \hlkwc{virginica} \hlstd{=} \hlstr{"typeC"} \hlstd{)}
\hlstd{> }\hlkwd{sample}\hlstd{(a[iris}\hlopt{$}\hlstd{Species],}\hlnum{5}\hlstd{)}
\end{alltt}
\end{kframe}
\end{knitrout}

\section{Petit exercice 5}

\begin{knitrout}\footnotesize
\definecolor{shadecolor}{rgb}{0.969, 0.969, 0.969}\color{fgcolor}\begin{kframe}
\begin{alltt}
\hlstd{> }\hlstd{a} \hlkwb{=} \hlkwd{c}\hlstd{(} \hlkwc{setosa} \hlstd{=} \hlstr{"typeA"}\hlstd{,} \hlkwc{versicolor} \hlstd{=} \hlstr{"typeB"}\hlstd{,}
\hlstd{+ }       \hlkwc{virginica} \hlstd{=} \hlstr{"typeC"} \hlstd{)}
\hlstd{> }\hlkwd{sample}\hlstd{(a[iris}\hlopt{$}\hlstd{Species],}\hlnum{5}\hlstd{)}
\end{alltt}
\end{kframe}
\end{knitrout}

\end{document}
