\documentclass{beamer}\usepackage[]{graphicx}\usepackage[]{color}
%% maxwidth is the original width if it is less than linewidth
%% otherwise use linewidth (to make sure the graphics do not exceed the margin)
\makeatletter
\def\maxwidth{ %
  \ifdim\Gin@nat@width>\linewidth
    \linewidth
  \else
    \Gin@nat@width
  \fi
}
\makeatother

\definecolor{fgcolor}{rgb}{0.345, 0.345, 0.345}
\newcommand{\hlnum}[1]{\textcolor[rgb]{0.686,0.059,0.569}{#1}}%
\newcommand{\hlstr}[1]{\textcolor[rgb]{0.192,0.494,0.8}{#1}}%
\newcommand{\hlcom}[1]{\textcolor[rgb]{0.678,0.584,0.686}{\textit{#1}}}%
\newcommand{\hlopt}[1]{\textcolor[rgb]{0,0,0}{#1}}%
\newcommand{\hlstd}[1]{\textcolor[rgb]{0.345,0.345,0.345}{#1}}%
\newcommand{\hlkwa}[1]{\textcolor[rgb]{0.161,0.373,0.58}{\textbf{#1}}}%
\newcommand{\hlkwb}[1]{\textcolor[rgb]{0.69,0.353,0.396}{#1}}%
\newcommand{\hlkwc}[1]{\textcolor[rgb]{0.333,0.667,0.333}{#1}}%
\newcommand{\hlkwd}[1]{\textcolor[rgb]{0.737,0.353,0.396}{\textbf{#1}}}%

\usepackage{framed}
\makeatletter
\newenvironment{kframe}{%
 \def\at@end@of@kframe{}%
 \ifinner\ifhmode%
  \def\at@end@of@kframe{\end{minipage}}%
  \begin{minipage}{\columnwidth}%
 \fi\fi%
 \def\FrameCommand##1{\hskip\@totalleftmargin \hskip-\fboxsep
 \colorbox{shadecolor}{##1}\hskip-\fboxsep
     % There is no \\@totalrightmargin, so:
     \hskip-\linewidth \hskip-\@totalleftmargin \hskip\columnwidth}%
 \MakeFramed {\advance\hsize-\width
   \@totalleftmargin\z@ \linewidth\hsize
   \@setminipage}}%
 {\par\unskip\endMakeFramed%
 \at@end@of@kframe}
\makeatother

\definecolor{shadecolor}{rgb}{.97, .97, .97}
\definecolor{messagecolor}{rgb}{0, 0, 0}
\definecolor{warningcolor}{rgb}{1, 0, 1}
\definecolor{errorcolor}{rgb}{1, 0, 0}
\newenvironment{knitrout}{}{} % an empty environment to be redefined in TeX

\usepackage{alltt}
\usetheme[compress]{Singapore}
\useoutertheme{miniframes}

% \documentclass{beamer}
%\usetheme{Warsaw}

% Pour les documents en francais...
	\usepackage[latin1]{inputenc}
	\usepackage[french]{babel}
	\usepackage[french]{varioref}

% Math?matiques
	\usepackage{amsmath}

% Caracteres speciaux suppl?mentaires
	\usepackage{latexsym,amsfonts}

% A documenter
	\usepackage{moreverb}

% Macros pour les paquets
	\usepackage{array}  			% N?cessaires pour les tableaux de la macro Excel.

% Outil suppl?mentaire pour les tableaux
	\usepackage{multirow}
	\usepackage{booktabs}
	\usepackage{xcolor} % alternating row colors in table, incompatible avec certains modules
	\usepackage{longtable}
	\usepackage{colortbl}

% Pour ins?rer des graphiques
	\usepackage{graphicx} 			% Graphique simples
	\usepackage{subfigure}			% Graphiques multiples

% Pour ins?rer des couleurs
	\usepackage{color}

% Rotation des objets et des pages
%	\usepackage{rotating}
%	\usepackage{lscape}

% Pour insrer du code source, LaTeX ou SAS par exemple.
	\usepackage{verbatim}
        \usepackage{moreverb}
	\usepackage{listings}
	\usepackage{fancyvrb}

%	\lstset{language=SAS,numbers=left}		% Par dfaut le listing est en SAS

% Pour ins?rer des hyperliens
  \usepackage{hyperref}

% American Psychological Association (for bibliographic references).
	\usepackage{apacite}

% Pour l'utilisation des macros
	\usepackage{xspace}

% Pour l'utilisation de notes en fin de document.
%	\usepackage{endnotes}

% Array
%	\usepackage{multirow}
%	\usepackage{booktabs}

% Rotation
%	\usepackage{rotating}

% En t?tes et pieds de pages
%	\usepackage{fancyhdr}
%	\usepackage{lastpage}


% Page layout

% By LaTeX commands
%\setlength{\oddsidemargin}{0cm}
%\setlength{\textwidth}{16cm}
%\setlength{\textheight}{24cm}
%\setlength{\topmargin}{-1cm}
%\setlength{\marginparsep}{0.2cm}

% fancyheader parameters
%\pagestyle{fancy}

%\fancyfoot[L]{{\small Formation \LaTeX, DEPP}}
%\fancyfoot[c]{}
%\fancyfoot[R]{{\small \thepage/\pageref{LastPage}}}

%\fancyhead[L]{}
%\fancyhead[c]{}
%\fancyhead[R]{}

% Pour ins?rer des dessins de Linux
\newcommand{\LinuxA}{\includegraphics[height=0.5cm]{Graphiques/linux.png}}
\newcommand{\LinuxB}{\includegraphics[height=0.5cm]{Graphiques/linux.png}\xspace}

% Macro pour les petits dessins pour les diff?rents OS.
\newcommand{\Windows}{\emph{Windows}\xspace}
\newcommand{\Mac}{\emph{Mac OS X}\xspace}
\newcommand{\Linux}{\emph{Linux}\xspace}
\newcommand{\MikTeX}{MiK\tex\xspace}
\newcommand{\latex}{\LaTeX\xspace}


\newcommand{\df}{\emph{data.frame}\xspace}
\newcommand{\liste}{\emph{list}\xspace}
\newcommand{\cad}{c'est-�-dire\xspace}

% Titre
\title{Introduction � R}
\author{Pascal Bessonneau}
%\institute{DEPP}
\date{05/2016}
\subtitle{Exercices, Simulation}


\newcommand{\hreff}[2]{\underline{\href{#1}{#2}\xspace}}



\IfFileExists{upquote.sty}{\usepackage{upquote}}{}
\begin{document}

\begin{frame}
	\maketitle
\end{frame}

\begin{frame}
	\tableofcontents
\end{frame}

\section{Exercices sur la loi normale}

\begin{frame}[containsverbatim]
  \frametitle{La loi normale}

  La fonction \emph{rnorm} permet de cr�er un vecteur contenant des nombres tir�s
  de la loi normale~:

\begin{knitrout}\footnotesize
\definecolor{shadecolor}{rgb}{0.969, 0.969, 0.969}\color{fgcolor}\begin{kframe}
\begin{alltt}
\hlstd{> }\hlstd{a} \hlkwb{<-} \hlkwd{rnorm}\hlstd{(}\hlnum{10000}\hlstd{)}
\hlstd{> }\hlkwd{plot}\hlstd{(a)}
\end{alltt}
\end{kframe}
\end{knitrout}

\end{frame}


\begin{frame}[containsverbatim]
  \frametitle{La loi normale}

  La fonction \emph{rnorm} permet de cr�er un vecteur contenant des nombres tir�s
  de la loi normale~:

\begin{knitrout}\footnotesize
\definecolor{shadecolor}{rgb}{0.969, 0.969, 0.969}\color{fgcolor}\begin{kframe}
\begin{alltt}
\hlstd{> }\hlkwd{hist}\hlstd{(a)}
\end{alltt}
\end{kframe}
\end{knitrout}

\end{frame}

\begin{frame}[containsverbatim]
  \frametitle{La loi normale}

  La fonction \emph{rnorm} permet de cr�er un vecteur contenant des nombres tir�s
  de la loi normale~:

\begin{knitrout}\footnotesize
\definecolor{shadecolor}{rgb}{0.969, 0.969, 0.969}\color{fgcolor}\begin{kframe}
\begin{alltt}
\hlstd{> }\hlstd{a} \hlkwb{<-} \hlkwd{rnorm}\hlstd{(}\hlnum{10000}\hlstd{,}\hlkwc{mean}\hlstd{=}\hlnum{2}\hlstd{,}\hlkwc{sd}\hlstd{=}\hlnum{4}\hlstd{)}
\end{alltt}
\end{kframe}
\end{knitrout}

\end{frame}

\begin{frame}[containsverbatim]
  \frametitle{La loi normale}

  La fonction \emph{rnorm} permet de cr�er un vecteur contenant des nombres tir�s
  de la loi normale~:

\begin{knitrout}\footnotesize
\definecolor{shadecolor}{rgb}{0.969, 0.969, 0.969}\color{fgcolor}\begin{kframe}
\begin{alltt}
\hlstd{> }\hlkwd{hist}\hlstd{(a)}
\end{alltt}
\end{kframe}
\end{knitrout}

\end{frame}

\section{Simulation de tirage de d�s}

\begin{frame}[containsverbatim]
  \frametitle{Les d�s}

  On peut aisement sous R simuler un tirage d'une pi�ce~:
  
\begin{knitrout}\footnotesize
\definecolor{shadecolor}{rgb}{0.969, 0.969, 0.969}\color{fgcolor}\begin{kframe}
\begin{alltt}
\hlstd{> }\hlstd{a} \hlkwb{<-} \hlkwd{rbinom}\hlstd{(}\hlnum{10000}\hlstd{,}\hlkwc{size} \hlstd{=} \hlnum{1}\hlstd{,} \hlkwc{prob} \hlstd{=} \hlnum{1} \hlopt{/} \hlnum{2} \hlstd{)}
\hlstd{> }\hlkwd{mean}\hlstd{(a)}
\hlstd{> }\hlkwd{sd}\hlstd{(a)}
\hlstd{> }\hlkwd{table}\hlstd{(a)}
\end{alltt}
\end{kframe}
\end{knitrout}
\end{frame}

\begin{frame}[containsverbatim]
  \frametitle{La convergence vers la loi normale}

  On peut simuler k tirages de n pi�ces~:
  
\begin{knitrout}\footnotesize
\definecolor{shadecolor}{rgb}{0.969, 0.969, 0.969}\color{fgcolor}\begin{kframe}
\begin{alltt}
\hlstd{> }\hlstd{k} \hlkwb{<-} \hlnum{10}
\hlstd{> }\hlstd{n} \hlkwb{<-} \hlnum{10}
\hlstd{> }\hlstd{m} \hlkwb{<-} \hlkwd{c}\hlstd{()}
\hlstd{> }\hlkwa{for} \hlstd{(ii} \hlkwa{in} \hlnum{1}\hlopt{:}\hlstd{k) \{}
\hlstd{+ }  \hlstd{m} \hlkwb{<-} \hlkwd{c}\hlstd{( m,} \hlkwd{mean}\hlstd{(} \hlkwd{rbinom}\hlstd{(n,} \hlkwc{size} \hlstd{=} \hlnum{1}\hlstd{,} \hlkwc{prob} \hlstd{=} \hlnum{1}\hlopt{/}\hlnum{2} \hlstd{) ) )}
\hlstd{+ }\hlstd{\}}
\hlstd{> }\hlstd{m}
\end{alltt}
\end{kframe}
\end{knitrout}
\end{frame}

\begin{frame}[containsverbatim]
  \frametitle{La convergence vers la loi normale}

\begin{knitrout}\footnotesize
\definecolor{shadecolor}{rgb}{0.969, 0.969, 0.969}\color{fgcolor}\begin{kframe}
\begin{alltt}
\hlstd{> }\hlkwd{hist}\hlstd{(m)}
\end{alltt}
\end{kframe}
\end{knitrout}
\end{frame}

\begin{frame}[containsverbatim]
  \frametitle{La convergence vers la loi normale}

\begin{knitrout}\footnotesize
\definecolor{shadecolor}{rgb}{0.969, 0.969, 0.969}\color{fgcolor}\begin{kframe}
\begin{alltt}
\hlstd{> }\hlstd{k} \hlkwb{<-} \hlnum{10000}
\hlstd{> }\hlstd{n} \hlkwb{<-} \hlnum{10}
\hlstd{> }\hlstd{m} \hlkwb{<-} \hlkwd{c}\hlstd{()}
\hlstd{> }\hlkwa{for} \hlstd{(ii} \hlkwa{in} \hlnum{1}\hlopt{:}\hlstd{k) \{}
\hlstd{+ }  \hlstd{m} \hlkwb{<-} \hlkwd{c}\hlstd{( m,} \hlkwd{mean}\hlstd{(}\hlkwd{rbinom}\hlstd{(}\hlnum{10}\hlstd{,}\hlkwc{size} \hlstd{=} \hlnum{1}\hlstd{,} \hlkwc{prob} \hlstd{=} \hlnum{1} \hlopt{/} \hlnum{2} \hlstd{)) )}
\hlstd{+ }\hlstd{\}}
\end{alltt}
\end{kframe}
\end{knitrout}

\end{frame}

\begin{frame}[containsverbatim]
  \frametitle{La convergence vers la loi normale}

\begin{knitrout}\footnotesize
\definecolor{shadecolor}{rgb}{0.969, 0.969, 0.969}\color{fgcolor}\begin{kframe}
\begin{alltt}
\hlstd{> }\hlkwd{hist}\hlstd{(m)}
\end{alltt}
\end{kframe}
\end{knitrout}

\end{frame}

\begin{frame}[containsverbatim]
  \frametitle{La convergence vers la loi normale}

  On dit en statistiques qu'il y a convergence vers la loi normale.
  
  Quand on tire une pi�ce, la valeur obtenu suit une loi binomiale. C'est li� � la
  m�thode de tirage et au \emph{design} de l'exp�rience. 
  
  Cette moyenne varie car on tire un �chantillon des valeurs possibles. En cons�quence
  la moyenne obtenue va varier pour chaque �chantillon. Et la valeur de la moyenne
  va varier en suivant une loi normale.  
  
\end{frame}


\section{Coding Challenge}

\begin{frame}[containsverbatim]
  \frametitle{Simuler une matrice de k x k nombres al�atoires}
  
  la loi peut �tre la loi uniforme, la loi normale, \dots
  
  Le but, obtenir une matrice de k x k avec $k^2$ nombres al�atoires.
  
  Etonnez moi ! 
  
  Un conseil commencez avec k �gal 10 ou 50. Puis avec k �gal � 1000.
  
\end{frame}


\begin{frame}[containsverbatim]
  \frametitle{Simuler une matrice de k x k nombres al�atoires}
  
  Pour cr�er une matrice la commande est~:
  
\begin{knitrout}\footnotesize
\definecolor{shadecolor}{rgb}{0.969, 0.969, 0.969}\color{fgcolor}\begin{kframe}
\begin{alltt}
\hlstd{> }\hlkwd{matrix}\hlstd{(valeurs,}\hlkwc{nrow}\hlstd{=k,}\hlkwc{ncol}\hlstd{=k)}
\end{alltt}
\end{kframe}
\end{knitrout}

\end{frame}


\begin{frame}[containsverbatim]
  \frametitle{Proposition 1}



  
\begin{knitrout}\footnotesize
\definecolor{shadecolor}{rgb}{0.969, 0.969, 0.969}\color{fgcolor}\begin{kframe}
\begin{alltt}
\hlstd{> }\hlstd{matrice} \hlkwb{<-} \hlkwd{matrix}\hlstd{(}\hlnum{NA}\hlstd{,}\hlkwc{nrow}\hlstd{=k,}\hlkwc{ncol}\hlstd{=k)}
\hlstd{> }\hlkwa{for} \hlstd{(ii} \hlkwa{in} \hlnum{1}\hlopt{:}\hlstd{k) \{}
\hlstd{+ }  \hlkwa{for} \hlstd{(jj} \hlkwa{in} \hlnum{1}\hlopt{:}\hlstd{k) \{}
\hlstd{+ }    \hlstd{matrice[ii,jj]} \hlkwb{<-} \hlkwd{rnorm}\hlstd{(}\hlnum{1}\hlstd{)}
\hlstd{+ }  \hlstd{\}}
\hlstd{+ }\hlstd{\}}
\end{alltt}
\end{kframe}
\end{knitrout}

\end{frame}

\begin{frame}[containsverbatim]
  \frametitle{Proposition 2}
  
\begin{knitrout}\footnotesize
\definecolor{shadecolor}{rgb}{0.969, 0.969, 0.969}\color{fgcolor}\begin{kframe}
\begin{alltt}
\hlstd{> }\hlstd{matrice} \hlkwb{<-} \hlkwd{matrix}\hlstd{(}\hlnum{NA}\hlstd{,}\hlkwc{nrow}\hlstd{=k,}\hlkwc{ncol}\hlstd{=k)}
\hlstd{> }\hlkwa{for} \hlstd{(ii} \hlkwa{in} \hlnum{1}\hlopt{:}\hlstd{k) \{}
\hlstd{+ }  \hlstd{matrice[ii,]} \hlkwb{<-} \hlkwd{rnorm}\hlstd{(k)}
\hlstd{+ }\hlstd{\}}
\end{alltt}
\end{kframe}
\end{knitrout}

\end{frame}

\begin{frame}[containsverbatim]
  \frametitle{Proposition 3}
  
  
\begin{knitrout}\footnotesize
\definecolor{shadecolor}{rgb}{0.969, 0.969, 0.969}\color{fgcolor}\begin{kframe}
\begin{alltt}
\hlstd{> }\hlstd{matrice} \hlkwb{<-} \hlkwd{matrix}\hlstd{(}\hlnum{1}\hlstd{,}\hlkwc{nrow}\hlstd{=k,}\hlkwc{ncol}\hlstd{=k)}
\hlstd{> }\hlstd{matrice} \hlkwb{<-} \hlkwd{apply}\hlstd{(matrice,}\hlnum{1}\hlopt{:}\hlnum{2}\hlstd{,rnorm)}
\end{alltt}
\end{kframe}
\end{knitrout}

\end{frame}


\begin{frame}[containsverbatim]
  \frametitle{Proposition 4}
  
\begin{knitrout}\footnotesize
\definecolor{shadecolor}{rgb}{0.969, 0.969, 0.969}\color{fgcolor}\begin{kframe}
\begin{alltt}
\hlstd{> }\hlstd{matrice} \hlkwb{<-} \hlkwd{rnorm}\hlstd{(k}\hlopt{^}\hlnum{2}\hlstd{)}
\hlstd{> }\hlkwd{dim}\hlstd{(matrice)} \hlkwb{<-} \hlkwd{c}\hlstd{(k,k)}
\hlstd{> }
\hlstd{> }\hlcom{# je sais, c'est pas esth�tique.}
\end{alltt}
\end{kframe}
\end{knitrout}

\end{frame}

\begin{frame}[containsverbatim]
  \frametitle{Mouvements brownien}
  
Random walking: un homme avance de pas en pas � chaque fois d'une valeur
tir� dans une loi normale dans la direction x et idem pour la direction y.
R�aliser les simulations pour n points

\end{frame}


\begin{frame}[containsverbatim]
  \frametitle{Mouvements brownien}
  
\begin{knitrout}\footnotesize
\definecolor{shadecolor}{rgb}{0.969, 0.969, 0.969}\color{fgcolor}\begin{kframe}
\begin{alltt}
\hlstd{> }\hlstd{n} \hlkwb{<-} \hlnum{10000}
\hlstd{> }\hlstd{position.x} \hlkwb{<-} \hlkwd{c}\hlstd{(}\hlnum{0}\hlstd{)}
\hlstd{> }\hlstd{position.y} \hlkwb{<-} \hlkwd{c}\hlstd{(}\hlnum{0}\hlstd{)}
\hlstd{> }\hlkwa{for} \hlstd{(ii} \hlkwa{in} \hlnum{1}\hlopt{:}\hlstd{n) \{}
\hlstd{+ }  \hlstd{position.x} \hlkwb{<-} \hlkwd{c}\hlstd{(position.x,position.x[ii}\hlopt{-}\hlnum{1}\hlstd{]}\hlopt{+}\hlkwd{rnorm}\hlstd{(}\hlnum{1}\hlstd{))}
\hlstd{+ }  \hlstd{position.y} \hlkwb{<-} \hlkwd{c}\hlstd{(position.y,position.y[ii}\hlopt{-}\hlnum{1}\hlstd{]}\hlopt{+}\hlkwd{rnorm}\hlstd{(}\hlnum{1}\hlstd{))}
\hlstd{+ }\hlstd{\}}
\hlstd{> }\hlkwd{plot}\hlstd{(}\hlnum{0}\hlstd{,}\hlnum{0}\hlstd{,}\hlkwc{type}\hlstd{=}\hlstr{"n"}\hlstd{,}\hlkwc{xlim}\hlstd{=}\hlkwd{range}\hlstd{(position.x),}\hlkwc{ylim}\hlstd{=}\hlkwd{range}\hlstd{(position.y))}
\hlstd{> }\hlkwd{lines}\hlstd{(position.x,position.y)}
\end{alltt}
\end{kframe}
\end{knitrout}

\end{frame}

\begin{frame}[containsverbatim]
  \frametitle{Mouvements brownien}
  
\begin{knitrout}\footnotesize
\definecolor{shadecolor}{rgb}{0.969, 0.969, 0.969}\color{fgcolor}\begin{kframe}
\begin{alltt}
\hlstd{> }\hlstd{n} \hlkwb{<-} \hlnum{10000}
\hlstd{> }\hlstd{position.x} \hlkwb{<-} \hlkwd{cumsum}\hlstd{(}\hlkwd{c}\hlstd{(} \hlnum{0}\hlstd{,} \hlkwd{rnorm}\hlstd{(n)))}
\hlstd{> }\hlstd{position.y} \hlkwb{<-} \hlkwd{cumsum}\hlstd{(}\hlkwd{c}\hlstd{(} \hlnum{0}\hlstd{,} \hlkwd{rnorm}\hlstd{(n)))}
\hlstd{> }\hlkwd{plot}\hlstd{(}\hlnum{0}\hlstd{,}\hlnum{0}\hlstd{,}\hlkwc{type}\hlstd{=}\hlstr{"n"}\hlstd{,}\hlkwc{xlim}\hlstd{=}\hlkwd{range}\hlstd{(position.x),}\hlkwc{ylim}\hlstd{=}\hlkwd{range}\hlstd{(position.y))}
\hlstd{> }\hlkwd{lines}\hlstd{(position.x,position.y)}
\end{alltt}
\end{kframe}
\end{knitrout}

\end{frame}






\end{document}
