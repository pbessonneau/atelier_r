\documentclass{article}\usepackage[]{graphicx}\usepackage[]{color}
%% maxwidth is the original width if it is less than linewidth
%% otherwise use linewidth (to make sure the graphics do not exceed the margin)
\makeatletter
\def\maxwidth{ %
  \ifdim\Gin@nat@width>\linewidth
    \linewidth
  \else
    \Gin@nat@width
  \fi
}
\makeatother

\definecolor{fgcolor}{rgb}{0.345, 0.345, 0.345}
\newcommand{\hlnum}[1]{\textcolor[rgb]{0.686,0.059,0.569}{#1}}%
\newcommand{\hlstr}[1]{\textcolor[rgb]{0.192,0.494,0.8}{#1}}%
\newcommand{\hlcom}[1]{\textcolor[rgb]{0.678,0.584,0.686}{\textit{#1}}}%
\newcommand{\hlopt}[1]{\textcolor[rgb]{0,0,0}{#1}}%
\newcommand{\hlstd}[1]{\textcolor[rgb]{0.345,0.345,0.345}{#1}}%
\newcommand{\hlkwa}[1]{\textcolor[rgb]{0.161,0.373,0.58}{\textbf{#1}}}%
\newcommand{\hlkwb}[1]{\textcolor[rgb]{0.69,0.353,0.396}{#1}}%
\newcommand{\hlkwc}[1]{\textcolor[rgb]{0.333,0.667,0.333}{#1}}%
\newcommand{\hlkwd}[1]{\textcolor[rgb]{0.737,0.353,0.396}{\textbf{#1}}}%

\usepackage{framed}
\makeatletter
\newenvironment{kframe}{%
 \def\at@end@of@kframe{}%
 \ifinner\ifhmode%
  \def\at@end@of@kframe{\end{minipage}}%
  \begin{minipage}{\columnwidth}%
 \fi\fi%
 \def\FrameCommand##1{\hskip\@totalleftmargin \hskip-\fboxsep
 \colorbox{shadecolor}{##1}\hskip-\fboxsep
     % There is no \\@totalrightmargin, so:
     \hskip-\linewidth \hskip-\@totalleftmargin \hskip\columnwidth}%
 \MakeFramed {\advance\hsize-\width
   \@totalleftmargin\z@ \linewidth\hsize
   \@setminipage}}%
 {\par\unskip\endMakeFramed%
 \at@end@of@kframe}
\makeatother

\definecolor{shadecolor}{rgb}{.97, .97, .97}
\definecolor{messagecolor}{rgb}{0, 0, 0}
\definecolor{warningcolor}{rgb}{1, 0, 1}
\definecolor{errorcolor}{rgb}{1, 0, 0}
\newenvironment{knitrout}{}{} % an empty environment to be redefined in TeX

\usepackage{alltt}
% \usetheme[compress]{Singapore}
% \useoutertheme{miniframes}

% \documentclass{beamer}
%\usetheme{Warsaw}

% Pour les documents en francais...
	\usepackage[latin1]{inputenc}
	\usepackage[french]{babel}
	\usepackage[french]{varioref}

% Math?matiques
	\usepackage{amsmath}

% Caracteres speciaux suppl?mentaires
	\usepackage{latexsym,amsfonts}

% A documenter
	\usepackage{moreverb}

% Macros pour les paquets
	\usepackage{array}  			% N?cessaires pour les tableaux de la macro Excel.

% Outil suppl?mentaire pour les tableaux
	\usepackage{multirow}
	\usepackage{booktabs}
	\usepackage{xcolor} % alternating row colors in table, incompatible avec certains modules
	\usepackage{longtable}
	\usepackage{colortbl}

% Pour ins?rer des graphiques
	\usepackage{graphicx} 			% Graphique simples
	\usepackage{subfigure}			% Graphiques multiples

% Pour ins?rer des couleurs
	\usepackage{color}

% Rotation des objets et des pages
%	\usepackage{rotating}
%	\usepackage{lscape}

% Pour insrer du code source, LaTeX ou SAS par exemple.
	\usepackage{verbatim}
        \usepackage{moreverb}
	\usepackage{listings}
	\usepackage{fancyvrb}

%	\lstset{language=SAS,numbers=left}		% Par dfaut le listing est en SAS

% Pour ins?rer des hyperliens
  \usepackage{hyperref}

% American Psychological Association (for bibliographic references).
	\usepackage{apacite}

% Pour l'utilisation des macros
	\usepackage{xspace}

% Pour l'utilisation de notes en fin de document.
%	\usepackage{endnotes}

% Array
%	\usepackage{multirow}
%	\usepackage{booktabs}

% Rotation
%	\usepackage{rotating}

% En t?tes et pieds de pages
%	\usepackage{fancyhdr}
%	\usepackage{lastpage}


% Page layout

% By LaTeX commands
%\setlength{\oddsidemargin}{0cm}
%\setlength{\textwidth}{16cm}
%\setlength{\textheight}{24cm}
%\setlength{\topmargin}{-1cm}
%\setlength{\marginparsep}{0.2cm}

% fancyheader parameters
%\pagestyle{fancy}

%\fancyfoot[L]{{\small Formation \LaTeX, DEPP}}
%\fancyfoot[c]{}
%\fancyfoot[R]{{\small \thepage/\pageref{LastPage}}}

%\fancyhead[L]{}
%\fancyhead[c]{}
%\fancyhead[R]{}

% Pour ins?rer des dessins de Linux
\newcommand{\LinuxA}{\includegraphics[height=0.5cm]{Graphiques/linux.png}}
\newcommand{\LinuxB}{\includegraphics[height=0.5cm]{Graphiques/linux.png}\xspace}

% Macro pour les petits dessins pour les diff?rents OS.
\newcommand{\Windows}{\emph{Windows}\xspace}
\newcommand{\Mac}{\emph{Mac OS X}\xspace}
\newcommand{\Linux}{\emph{Linux}\xspace}
\newcommand{\MikTeX}{MiK\tex\xspace}
\newcommand{\latex}{\LaTeX\xspace}


\newcommand{\df}{\emph{data.frame}\xspace}
\newcommand{\liste}{\emph{list}\xspace}
\newcommand{\cad}{c'est-�-dire\xspace}

% Titre
\title{Introduction � R}
\author{Pascal Bessonneau}
%\institute{DEPP}
\date{06/2015}
%\subtitle{Les premi�res donn�es}


\newcommand{\hreff}[2]{\underline{\href{#1}{#2}\xspace}}



\IfFileExists{upquote.sty}{\usepackage{upquote}}{}
\begin{document}


\section{Les trac�s pour chaque �pid�mie en 2014/2015}


\section{Superposition des �pisodes �pid�miques}

Les questions de cet exercice sont les suivantes~:

\begin{enumerate}
  \item Importer les donn�es "Openhealth" pour les trois maladies
  \item Fusionner les tableaux pour l'ann�e pour n'en faire plus qu'un avec 
  une variable par maladie (seulement pour l'ann�e 2014/2015).
  \item La fusion va permettre de repr�senter graphiquement l'incidence des
  maladies sur un m�me graphique mois par mois
\end{enumerate}
  
  
\begin{knitrout}\footnotesize
\definecolor{shadecolor}{rgb}{0.969, 0.969, 0.969}\color{fgcolor}\begin{kframe}
\begin{alltt}
\hlstd{> }\hlstd{gale} \hlkwb{<-} \hlkwd{read.csv2}\hlstd{(}\hlstr{"data/maladies/Openhealth_Gale_Regions.csv"}\hlstd{,}\hlkwc{stringsAsFactors} \hlstd{= F)}
\hlstd{> }\hlstd{poux} \hlkwb{<-} \hlkwd{read.csv2}\hlstd{(}\hlstr{"data/maladies/Openhealth_poux_Regions.csv"}\hlstd{,}\hlkwc{stringsAsFactors} \hlstd{= F)}
\hlstd{> }\hlstd{grippe} \hlkwb{<-} \hlkwd{read.csv2}\hlstd{(}\hlstr{"data/maladies/Openhealth_S-Grippal_Regions.csv"}\hlstd{,}\hlkwc{stringsAsFactors} \hlstd{= F)}
\end{alltt}
\end{kframe}
\end{knitrout}

\begin{knitrout}\footnotesize
\definecolor{shadecolor}{rgb}{0.969, 0.969, 0.969}\color{fgcolor}\begin{kframe}
\begin{alltt}
\hlstd{> }\hlstd{gale}\hlopt{$}\hlstd{gale_2014_2015} \hlkwb{<-} \hlstd{gale}\hlopt{$}\hlstd{Sais_2014_2015}
\hlstd{> }\hlstd{poux}\hlopt{$}\hlstd{poux_2014_2015} \hlkwb{<-} \hlstd{poux}\hlopt{$}\hlstd{Sais_2014_2015}
\hlstd{> }\hlstd{grippe}\hlopt{$}\hlstd{grippe_2014_2015} \hlkwb{<-} \hlstd{grippe}\hlopt{$}\hlstd{Sais_2014_2015}
\end{alltt}
\end{kframe}
\end{knitrout}

\begin{knitrout}\footnotesize
\definecolor{shadecolor}{rgb}{0.969, 0.969, 0.969}\color{fgcolor}\begin{kframe}
\begin{alltt}
\hlstd{> }\hlstd{fusion} \hlkwb{<-} \hlkwd{merge}\hlstd{(gale[,}\hlkwd{c}\hlstd{(}\hlstr{"PERIODE"}\hlstd{,}\hlstr{"gale_2014_2015"}\hlstd{)],poux[,}\hlkwd{c}\hlstd{(}\hlstr{"PERIODE"}\hlstd{,}\hlstr{"poux_2014_2015"}\hlstd{)],}\hlkwc{all}\hlstd{=T)}
\hlstd{> }\hlkwd{dim}\hlstd{(gale);}\hlkwd{dim}\hlstd{(poux);}\hlkwd{dim}\hlstd{(fusion)}
\end{alltt}
\end{kframe}
\end{knitrout}

\begin{knitrout}\footnotesize
\definecolor{shadecolor}{rgb}{0.969, 0.969, 0.969}\color{fgcolor}\begin{kframe}
\begin{alltt}
\hlstd{> }\hlstd{fusion} \hlkwb{<-} \hlkwd{merge}\hlstd{(fusion,grippe[,}\hlkwd{c}\hlstd{(}\hlstr{"PERIODE"}\hlstd{,}\hlstr{"grippe_2014_2015"}\hlstd{)],}\hlkwc{all}\hlstd{=T)}
\hlstd{> }\hlkwd{dim}\hlstd{(fusion);}\hlkwd{dim}\hlstd{(grippe);}\hlkwd{dim}\hlstd{(fusion)}
\end{alltt}
\end{kframe}
\end{knitrout}

\begin{knitrout}\footnotesize
\definecolor{shadecolor}{rgb}{0.969, 0.969, 0.969}\color{fgcolor}\begin{kframe}
\begin{alltt}
\hlstd{> }\hlstd{fusion} \hlkwb{<-} \hlstd{fusion[}\hlopt{!}\hlkwd{is.na}\hlstd{(fusion}\hlopt{$}\hlstd{gale_2014_2015),]}
\hlstd{> }
\hlstd{> }\hlstd{fusion}\hlopt{$}\hlstd{date} \hlkwb{<-} \hlkwd{dmy}\hlstd{(fusion}\hlopt{$}\hlstd{PERIODE)}
\end{alltt}
\end{kframe}
\end{knitrout}

\begin{knitrout}\footnotesize
\definecolor{shadecolor}{rgb}{0.969, 0.969, 0.969}\color{fgcolor}\begin{kframe}
\begin{alltt}
\hlstd{> }\hlstd{ylim} \hlkwb{<-} \hlkwd{range}\hlstd{(} \hlkwd{c}\hlstd{(fusion}\hlopt{$}\hlstd{gale_2014_2015,fusion}\hlopt{$}\hlstd{grippe_2014_2015,fusion}\hlopt{$}\hlstd{poux_2014_2015) )}
\hlstd{> }
\hlstd{> }\hlstd{fusion} \hlkwb{<-} \hlstd{fusion[}\hlkwd{order}\hlstd{(fusion}\hlopt{$}\hlstd{date),]}
\hlstd{> }
\hlstd{> }\hlkwd{plot}\hlstd{(fusion}\hlopt{$}\hlstd{date, fusion}\hlopt{$}\hlstd{gale_2014_2015,} \hlkwc{col} \hlstd{=} \hlstr{"black"}\hlstd{,} \hlkwc{type}\hlstd{=}\hlstr{"l"}\hlstd{,} \hlkwc{ylim}\hlstd{=ylim)}
\hlstd{> }\hlkwd{lines}\hlstd{(fusion}\hlopt{$}\hlstd{date, fusion}\hlopt{$}\hlstd{grippe_2014_2015,} \hlkwc{col} \hlstd{=} \hlstr{"red"}\hlstd{)}
\hlstd{> }\hlkwd{lines}\hlstd{(fusion}\hlopt{$}\hlstd{date, fusion}\hlopt{$}\hlstd{poux_2014_2015,} \hlkwc{col} \hlstd{=} \hlstr{"blue"}\hlstd{)}
\end{alltt}
\end{kframe}
\end{knitrout}


\section{Repr�sentation graphique des courbes �pid�miques de la grippe pour toutes les ann�es}

\subsection{Partie 1}

Les questions de cet exercice sont les suivantes~:

\begin{enumerate}
  \item Couper la variable PERIODE pour enlever l'ann�e
  \item Repr�senter deux ann�es sur un m�me graphique
  \item Faire une boucle pour repr�senter toutes les ann�es
\end{enumerate}
  


\begin{knitrout}\footnotesize
\definecolor{shadecolor}{rgb}{0.969, 0.969, 0.969}\color{fgcolor}\begin{kframe}
\begin{alltt}
\hlstd{> }\hlstd{grippe} \hlkwb{<-} \hlstd{grippe[}\hlopt{!}\hlkwd{is.na}\hlstd{(grippe}\hlopt{$}\hlstd{Sais_2014_2015),]}
\hlstd{> }\hlstd{grippe}\hlopt{$}\hlstd{date} \hlkwb{<-} \hlkwd{substr}\hlstd{(grippe}\hlopt{$}\hlstd{PERIODE,}\hlnum{1}\hlstd{,}\hlnum{5}\hlstd{)}
\hlstd{> }
\hlstd{> }\hlstd{vars} \hlkwb{<-} \hlkwd{c}\hlstd{(}\hlstr{"Sais_2014_2015"}\hlstd{,}\hlstr{"Sais_2013_2014"}\hlstd{,}\hlstr{"Sais_2012_2013"}\hlstd{,}\hlstr{"Sais_2011_2012"}\hlstd{,} \hlstr{"Sais_2010_2011"}\hlstd{)}
\hlstd{> }
\hlstd{> }\hlstd{ylim} \hlkwb{<-} \hlkwd{c}\hlstd{()}
\hlstd{> }\hlkwa{for} \hlstd{(var} \hlkwa{in} \hlstd{vars)}
\hlstd{+ }  \hlstd{ylim} \hlkwb{<-} \hlkwd{range}\hlstd{(}\hlkwd{c}\hlstd{(ylim,grippe[[var]]),}\hlkwc{na.rm}\hlstd{=T)}
\hlstd{> }
\hlstd{> }\hlstd{couleurs} \hlkwb{<-} \hlkwd{rev}\hlstd{(}\hlkwd{brewer.pal}\hlstd{(}\hlkwd{length}\hlstd{(vars),}\hlstr{"YlOrBr"}\hlstd{))}
\hlstd{> }
\hlstd{> }\hlkwd{plot}\hlstd{(}\hlnum{1}\hlopt{:}\hlkwd{nrow}\hlstd{(grippe),}\hlkwd{rep}\hlstd{(}\hlnum{0}\hlstd{,}\hlkwd{nrow}\hlstd{(grippe)),}\hlkwc{type}\hlstd{=}\hlstr{"n"}\hlstd{,}\hlkwc{ylab}\hlstd{=}\hlstr{"indice"}\hlstd{,}\hlkwc{xlab}\hlstd{=}\hlstr{"date"}\hlstd{,}\hlkwc{main}\hlstd{=}\hlstr{"Grippe"}\hlstd{,}\hlkwc{ylim}\hlstd{=ylim)}
\hlstd{> }
\hlstd{> }\hlkwa{for} \hlstd{(var} \hlkwa{in} \hlnum{1}\hlopt{:}\hlkwd{length}\hlstd{(vars))}
\hlstd{+ }  \hlkwd{lines}\hlstd{(grippe[[vars[var]]],}\hlkwc{col}\hlstd{=couleurs[[var]])}
\hlstd{> }
\hlstd{> }\hlkwd{legend}\hlstd{(}\hlstr{"topright"}\hlstd{,vars,}\hlkwc{pch}\hlstd{=}\hlnum{20}\hlstd{,}\hlkwc{col}\hlstd{=couleurs)}
\end{alltt}
\end{kframe}
\end{knitrout}


\section{Partie 2 (avanc�e)}

Le but est de rep�rer les valeurs moyennes et maximums pour chaque saisons.

Comme il y a plusieurs variables, on pr�f�re utiliser \emph{melt} pour n'avoir qu'une
variable. 

Puis on utilise \emph{ddply} pour calculer les statistiques pour chaque saison.

\begin{knitrout}\footnotesize
\definecolor{shadecolor}{rgb}{0.969, 0.969, 0.969}\color{fgcolor}\begin{kframe}
\begin{alltt}
\hlstd{> }\hlstd{grippe2} \hlkwb{<-} \hlstd{grippe}
\hlstd{> }
\hlstd{> }\hlstd{grippe2} \hlkwb{<-} \hlstd{grippe[,}\hlkwd{grep}\hlstd{(}\hlstr{"Sais_|date"}\hlstd{,}\hlkwd{colnames}\hlstd{(grippe2))]}
\hlstd{> }\hlkwd{colnames}\hlstd{(grippe2)}
\hlstd{> }
\hlstd{> }\hlstd{grippem} \hlkwb{<-} \hlkwd{melt}\hlstd{(grippe2,}\hlkwc{id.vars}\hlstd{=}\hlkwd{c}\hlstd{(}\hlstr{"date"}\hlstd{))}
\hlstd{> }
\hlstd{> }\hlkwd{colnames}\hlstd{(grippem)}
\hlstd{> }
\hlstd{> }\hlstd{grippem} \hlopt \hlkwd{group_by}\hlstd{(variable)} \hlopt
\hlstd{+ }  \hlkwd{summarize}\hlstd{(}
\hlstd{+ }      \hlkwc{mean}\hlstd{=}\hlkwd{mean}\hlstd{(value,}\hlkwc{na.rm}\hlstd{=T),}
\hlstd{+ }      \hlkwc{sd}\hlstd{=}\hlkwd{sd}\hlstd{(value,}\hlkwc{na.rm}\hlstd{=T),}
\hlstd{+ }      \hlkwc{max}\hlstd{=}\hlkwd{max}\hlstd{(value,}\hlkwc{na.rm}\hlstd{=T)}
\hlstd{+ }  \hlstd{)}
\end{alltt}
\end{kframe}
\end{knitrout}

\end{document}
