% knitr

\usepackage[]{graphicx}
\usepackage[]{color}
\makeatletter
\def\maxwidth{ %
  \ifdim\Gin@nat@width>\linewidth
    \linewidth
  \else
    \Gin@nat@width
  \fi
}
\makeatother

\definecolor{fgcolor}{rgb}{0.345, 0.345, 0.345}
\newcommand{\hlnum}[1]{\textcolor[rgb]{0.686,0.059,0.569}{#1}}%
\newcommand{\hlstr}[1]{\textcolor[rgb]{0.192,0.494,0.8}{#1}}%
\newcommand{\hlcom}[1]{\textcolor[rgb]{0.678,0.584,0.686}{\textit{#1}}}%
\newcommand{\hlopt}[1]{\textcolor[rgb]{0,0,0}{#1}}%
\newcommand{\hlstd}[1]{\textcolor[rgb]{0.345,0.345,0.345}{#1}}%
\newcommand{\hlkwa}[1]{\textcolor[rgb]{0.161,0.373,0.58}{\textbf{#1}}}%
\newcommand{\hlkwb}[1]{\textcolor[rgb]{0.69,0.353,0.396}{#1}}%
\newcommand{\hlkwc}[1]{\textcolor[rgb]{0.333,0.667,0.333}{#1}}%
\newcommand{\hlkwd}[1]{\textcolor[rgb]{0.737,0.353,0.396}{\textbf{#1}}}%

\usepackage{framed}
\makeatletter
\newenvironment{kframe}{%
 \def\at@end@of@kframe{}%
 \ifinner\ifhmode%
  \def\at@end@of@kframe{\end{minipage}}%
  \begin{minipage}{\columnwidth}%
 \fi\fi%
 \def\FrameCommand##1{\hskip\@totalleftmargin \hskip-\fboxsep
 \colorbox{shadecolor}{##1}\hskip-\fboxsep
     % There is no \\@totalrightmargin, so:
     \hskip-\linewidth \hskip-\@totalleftmargin \hskip\columnwidth}%
 \MakeFramed {\advance\hsize-\width
   \@totalleftmargin\z@ \linewidth\hsize
   \@setminipage}}%
 {\par\unskip\endMakeFramed%
 \at@end@of@kframe}
\makeatother

%	\usepackage{knitr}
	
\definecolor{shadecolor}{rgb}{.97, .97, .97}
\definecolor{messagecolor}{rgb}{0, 0, 0}
\definecolor{warningcolor}{rgb}{1, 0, 1}
\definecolor{errorcolor}{rgb}{1, 0, 0}
\newenvironment{knitrout}{}{} % an empty environment to be redefined in TeX
%fin knitr
	
\usepackage{alltt}	

	
	
	\usepackage{ifthen}
	
	
	
% Pour les documents en francais...
	\usepackage[latin1]{inputenc}
	\usepackage[french]{babel}
	\usepackage[french]{varioref}

% Math�matiques
	\usepackage{amsmath}

% Caracteres speciaux suppl�mentaires
	\usepackage{latexsym,amsfonts}

% A documenter
	\usepackage{moreverb}

% Macros pour les paquets
	\usepackage{array}  			% N�cessaires pour les tableaux de la macro Excel.
	\usepackage{dcolumn}
	
% Outil suppl�mentaire pour les tableaux
	\usepackage{multirow}
	\usepackage{booktabs}
%	\usepackage{xcolor} % alternating row colors in table, incompatible avec certains modules
	\usepackage{longtable}
	\usepackage{colortbl}

% Pour ins�rer des graphiques
	\usepackage{graphicx} 			% Graphique simples
	\usepackage{subfigure}			% Graphiques multiples

% Pour ins�rer des couleurs
	\usepackage{color}

% Rotation des objets et des pages
%	\usepackage{rotating}
%	\usepackage{lscape}

% Pour insrer du code source, LaTeX ou SAS par exemple.
	\usepackage{verbatim}
	\usepackage{listings}
	\usepackage{fancyvrb}

%	\lstset{language=SAS,numbers=left}		% Par dfaut le listing est en SAS

% Pour ins�rer des hyperliens
  \usepackage{hyperref}

% American Psychological Association (for bibliographic references).
	\usepackage{apacite}

% Pour l'utilisation des macros
	\usepackage{xspace}

% Pour l'utilisation de notes en fin de document.
%	\usepackage{endnotes}

% Array
%	\usepackage{multirow}
%	\usepackage{booktabs}

% Rotation
%	\usepackage{rotating}

% En t�tes et pieds de pages
	\usepackage{fancyhdr}
	\usepackage{lastpage}
	
	


