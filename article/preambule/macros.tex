


% Pour insérer des dessins de Linux
\newcommand{\LinuxA}{\includegraphics[height=0.5cm]{Graphiques/linux.png}}
\newcommand{\LinuxB}{\includegraphics[height=0.5cm]{Graphiques/linux.png}\xspace}


% Macro pour les petits dessins pour les diff�rents OS.
\newcommand{\Windows}{\emph{Windows}\xspace}
\newcommand{\Mac}{\emph{Mac OS X}\xspace}
\newcommand{\Linux}{\emph{Linux}\xspace}
\newcommand{\MikTeX}{MiK\tex\xspace}

\newcommand{\df}{\emph{data.frame}\xspace}
\newcommand{\dfs}{\emph{data.frames}\xspace}
\newcommand{\liste}{\emph{list}\xspace}
\newcommand{\listes}{\emph{lists}\xspace}

\newcommand{\factor}{\emph{factor}\xspace}
\newcommand{\character}{\emph{character}\xspace}
\newcommand{\logical}{\emph{logical}\xspace}

\newcommand{\cad}{c'est-�-dire\xspace}

\newcommand{\hreff}[2]{\underline{\href{#1}{#2}\xspace}}

% Des raccourcis pour les commandes \LaTeX, \TeX, ...
\newcommand{\latex}{\LaTeX\xspace}
\newcommand{\latexe}{\LaTeXe\xspace}
\newcommand{\tex}{\TeX\xspace}

\newcommand{\indexcom}[1]{\index{#1@\textsl{#1}}}
  
\newcommand{\R}{\includegraphics[scale=0.025]{Rlogo}}
\newcommand{\licence}{\includegraphics[scale=0.4]{licence}}
  
%\makeatletter
%
%\renewcommand\section{\@startsection {section}{1}{\z@}%
%   {-3.5ex \@plus -1ex \@minus -.2ex}%
%   {2.3ex \@plus.2ex}%
%   {\normalfont\Large\bfseries\textcolor{blue}}}
%   
%\renewcommand\subsection{\@startsection{subsection}{2}{\z@}%
%   {-3.25ex\@plus -1ex \@minus -.2ex}%
%   {1.5ex \@plus .2ex}%
%   {\normalfont\large\bfseries\textcolor{blue}}}
%   
%\renewcommand\subsubsection{\@startsection{subsubsection}{2}{\z@}%
%   {-3.25ex\@plus -1ex \@minus -.2ex}%
%   {1.5ex \@plus .2ex}%
%   {\normalfont\large\bfseries\textcolor{blue}}}
%
%\def\hlinewd#1{%
%\noalign{\ifnum0=`}\fi\hrule \@height #1 %
%\futurelet\reserved@a\@xhline}
%
%\makeatother
