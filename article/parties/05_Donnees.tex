\section{Les iris de Fisher}

  \subsection{Les iris de Fisher}

  Les iris de Fisher sont des donn�es tr�s connues dans le milieu des statisticiens.
  
  Ils consituent un jeu de donn�es sur lesquelles on utilise des m�thodes de 
  classification notamment.
  
  Ce sont les caract�ristiques morphologiques des feuilles pour quelques esp�ces
  d'Iris.
  



  
\begin{knitrout}\footnotesize
\definecolor{shadecolor}{rgb}{0.969, 0.969, 0.969}\color{fgcolor}\begin{kframe}
\begin{alltt}
\hlstd{> }\hlkwd{data}\hlstd{(iris)}
\hlstd{> }\hlkwd{class}\hlstd{(iris)}
\end{alltt}
\begin{verbatim}
## [1] "data.frame"
\end{verbatim}
\end{kframe}
\end{knitrout}




\begin{knitrout}\footnotesize
\definecolor{shadecolor}{rgb}{0.969, 0.969, 0.969}\color{fgcolor}\begin{kframe}
\begin{alltt}
\hlstd{> }\hlkwd{summary}\hlstd{(iris)}
\end{alltt}
\begin{verbatim}
##   Sepal.Length    Sepal.Width   
##  Min.   :4.300   Min.   :2.000  
##  1st Qu.:5.100   1st Qu.:2.800  
##  Median :5.800   Median :3.000  
##  Mean   :5.843   Mean   :3.057  
##  3rd Qu.:6.400   3rd Qu.:3.300  
##  Max.   :7.900   Max.   :4.400  
##   Petal.Length    Petal.Width   
##  Min.   :1.000   Min.   :0.100  
##  1st Qu.:1.600   1st Qu.:0.300  
##  Median :4.350   Median :1.300  
##  Mean   :3.758   Mean   :1.199  
##  3rd Qu.:5.100   3rd Qu.:1.800  
##  Max.   :6.900   Max.   :2.500  
##        Species  
##  setosa    :50  
##  versicolor:50  
##  virginica :50  
##                 
##                 
## 
\end{verbatim}
\end{kframe}
\end{knitrout}



\begin{knitrout}\footnotesize
\definecolor{shadecolor}{rgb}{0.969, 0.969, 0.969}\color{fgcolor}\begin{kframe}
\begin{alltt}
\hlstd{> }\hlkwd{table}\hlstd{(iris}\hlopt{$}\hlstd{Species)}
\end{alltt}
\begin{verbatim}
## 
##     setosa versicolor  virginica 
##         50         50         50
\end{verbatim}
\begin{alltt}
\hlstd{> }\hlkwd{prop.table}\hlstd{(}\hlkwd{table}\hlstd{(iris}\hlopt{$}\hlstd{Species))}
\end{alltt}
\begin{verbatim}
## 
##     setosa versicolor  virginica 
##  0.3333333  0.3333333  0.3333333
\end{verbatim}
\begin{alltt}
\hlstd{> }\hlkwd{prop.table}\hlstd{(}\hlkwd{table}\hlstd{(iris}\hlopt{$}\hlstd{Species))}\hlopt{*}\hlnum{100}
\end{alltt}
\begin{verbatim}
## 
##     setosa versicolor  virginica 
##   33.33333   33.33333   33.33333
\end{verbatim}
\end{kframe}
\end{knitrout}




\begin{knitrout}\footnotesize
\definecolor{shadecolor}{rgb}{0.969, 0.969, 0.969}\color{fgcolor}\begin{kframe}
\begin{alltt}
\hlstd{> }\hlkwd{tapply}\hlstd{(iris}\hlopt{$}\hlstd{Sepal.Length,iris}\hlopt{$}\hlstd{Species,mean)}
\end{alltt}
\begin{verbatim}
##     setosa versicolor  virginica 
##      5.006      5.936      6.588
\end{verbatim}
\end{kframe}
\end{knitrout}


\section{Les pr�noms}

  \subsection{Les pr�noms � Paris}

  Ce sont les pr�noms des nouveaux n�s � Paris. Ils viennent de 
  \href{http://opendata.paris.fr/explore/dataset/liste_des_prenoms_2004_a_2012/information/?disjunctive.prenoms&disjunctive.annee}{opendata.paris.fr}.
  
  Le but ici est de manipuler et d'extraire les donn�es. 
  


Pour charger le fichier~:

\begin{knitrout}\footnotesize
\definecolor{shadecolor}{rgb}{0.969, 0.969, 0.969}\color{fgcolor}\begin{kframe}
\begin{alltt}
\hlstd{> }\hlstd{prenoms} \hlkwb{<-} \hlkwd{read.csv2}\hlstd{(}\hlstr{"data/prenoms/liste_des_prenoms_2004_a_2012.csv"}\hlstd{,}
\hlstd{+ }                     \hlkwc{stringsAsFactors} \hlstd{= F,}\hlkwc{encoding} \hlstd{=} \hlstr{"UTF-8"}\hlstd{)}
\hlstd{> }\hlstd{p} \hlkwb{<-} \hlkwd{fromJSON}\hlstd{(}\hlstr{"data/prenoms/liste_des_prenoms_2004_a_2012.json"}\hlstd{)}\hlopt{$}\hlstd{fields}
\hlstd{> }
\hlstd{> }\hlkwd{colnames}\hlstd{(prenoms)}
\end{alltt}
\begin{verbatim}
## [1] "Prenoms" "Nombre"  "Sexe"    "Annee"
\end{verbatim}
\end{kframe}
\end{knitrout}
  


Pour r�cup�rer les pr�noms de 2004~:

\begin{knitrout}\footnotesize
\definecolor{shadecolor}{rgb}{0.969, 0.969, 0.969}\color{fgcolor}\begin{kframe}
\begin{alltt}
\hlstd{> }\hlstd{prenoms2004} \hlkwb{<-} \hlstd{prenoms[prenoms}\hlopt{$}\hlstd{Annee}\hlopt{==}\hlnum{2004}\hlstd{,]}
\end{alltt}
\end{kframe}
\end{knitrout}
  




Quel est le pr�nom le plus fr�quent ?

\begin{knitrout}\footnotesize
\definecolor{shadecolor}{rgb}{0.969, 0.969, 0.969}\color{fgcolor}\begin{kframe}
\begin{alltt}
\hlstd{> }\hlkwd{max}\hlstd{(prenoms}\hlopt{$}\hlstd{Nombre)}
\end{alltt}
\begin{verbatim}
## [1] 398
\end{verbatim}
\begin{alltt}
\hlstd{> }\hlstd{prenoms2004}\hlopt{$}\hlstd{Prenoms[prenoms2004}\hlopt{$}\hlstd{Nombre} \hlopt{==} \hlkwd{max}\hlstd{(prenoms2004}\hlopt{$}\hlstd{Nombre)]}
\end{alltt}
\begin{verbatim}
## [1] "Alexandre"
\end{verbatim}
\end{kframe}
\end{knitrout}
  


Quel est le pr�nom le moins fr�quent ?

\begin{knitrout}\footnotesize
\definecolor{shadecolor}{rgb}{0.969, 0.969, 0.969}\color{fgcolor}\begin{kframe}
\begin{alltt}
\hlstd{> }\hlstd{prenoms2004}\hlopt{$}\hlstd{Prenoms[prenoms2004}\hlopt{$}\hlstd{Nombre} \hlopt{==} \hlkwd{min}\hlstd{(prenoms2004}\hlopt{$}\hlstd{Nombre)]}
\end{alltt}
\end{kframe}
\end{knitrout}

\begin{knitrout}\footnotesize
\definecolor{shadecolor}{rgb}{0.969, 0.969, 0.969}\color{fgcolor}\begin{kframe}
\begin{verbatim}
##   [1] "Khalil"     "Leana"      "Loubna"    
##   [4] "Morgan"     "Natalia"    "Oussama"   
##   [7] "Safa"       "Sharon"     "Solenn"    
##  [10] "Sylvia"     "Viktor"     "Virgil"    
##  [13] "Wandrille"  "Warren"     "Alexane"   
##  [16] NA           "Camelia"    "Carl"      
##  [19] "Chanel"     "Filipe"     "Halima"    
##  [22] "Henry"      "Iban"       "Jawad"     
##  [25] "Josh"       "Adil"       "Bahia"     
##  [28] "Boubou"     "Clothilde"  "Dana"      
##  [31] "Daria"      "Gabriella"  "Harold"    
##  [34] "Hasna"      NA           "Latifa"    
##  [37] "Louka"      "Mory"       "Nesrine"   
##  [40] "Niouma"     "Rami"       "Ramy"      
##  [43] "Reda"       "Sebastian"  "Tim"       
##  [46] "Wendy"      NA           "Aboubakar" 
##  [49] "Adeline"    "Aymane"     "Benoit"    
##  [52] "Betty"      "Brune"      NA          
##  [55] "Colette"    "Cyriaque"   "Djeneba"   
##  [58] "Doriane"    "Elio"       "Germain"   
##  [61] "Guy"        "Ian"        "Idris"     
##  [64] "Ilyass"     "Khady"      "Nayla"     
##  [67] "Patricia"   "Sadio"      "Sylvain"   
##  [70] "Vladimir"   "Yanni"      "Khadija"   
##  [73] "Lamine"     "Lirone"     "Liza"      
##  [76] "Manuel"     "Naya"       "Nikita"    
##  [79] "Olympe"     "Perle"      "Solveig"   
##  [82] "Terence"    "Wilfried"   "Yossef"    
##  [85] NA           "Albert"     "Alissa"    
##  [88] "Aris"       "Calvin"     "Abdellah"  
##  [91] "Amara"      "Harry"      "Amelia"    
##  [94] "Athena"     "Brayan"     "Chelsea"   
##  [97] "Elena"      "Eliane"     "Elya"      
## [100] "Emy"        "Florence"   "Gad"       
## [103] NA           "Joey"       "Kadidiatou"
## [106] "Luce"       "Mahe"       "Meline"    
## [109] "Nael"       "Odelia"     "Oren"      
## [112] "Paco"       "Satine"     "Tao"       
## [115] NA
\end{verbatim}
\end{kframe}
\end{knitrout}
  



Quel est le minimum ?
\begin{knitrout}\footnotesize
\definecolor{shadecolor}{rgb}{0.969, 0.969, 0.969}\color{fgcolor}\begin{kframe}
\begin{alltt}
\hlstd{> }\hlkwd{min}\hlstd{(prenoms2004}\hlopt{$}\hlstd{Nombre)}
\end{alltt}
\begin{verbatim}
## [1] 6
\end{verbatim}
\end{kframe}
\end{knitrout}
  


Choisissez un pr�nom et trouver le nombre correspondants~:
\begin{knitrout}\footnotesize
\definecolor{shadecolor}{rgb}{0.969, 0.969, 0.969}\color{fgcolor}\begin{kframe}
\begin{alltt}
\hlstd{> }\hlstd{prenoms2004}\hlopt{$}\hlstd{Nombre[prenoms2004}\hlopt{$}\hlstd{Prenoms}\hlopt{==}\hlstr{"Pascal"}\hlstd{]}
\end{alltt}
\begin{verbatim}
## [1] 11
\end{verbatim}
\end{kframe}
\end{knitrout}
  


  Trouver le pr�noms qui ont disparus entre ces ann�es. Cela revient �
  faire une table et � chercher les pr�noms qui apparaissent moins de 8 fois.
  


\begin{knitrout}\footnotesize
\definecolor{shadecolor}{rgb}{0.969, 0.969, 0.969}\color{fgcolor}\begin{kframe}
\begin{alltt}
\hlstd{> }\hlstd{tt} \hlkwb{<-} \hlkwd{table}\hlstd{(prenoms}\hlopt{$}\hlstd{Prenoms)}
\hlstd{> }\hlkwd{range}\hlstd{(tt)}
\end{alltt}
\begin{verbatim}
## [1]  1 26
\end{verbatim}
\end{kframe}
\end{knitrout}



  Oups y'a une petit probl�me dans la base de donn�es.

\begin{knitrout}\footnotesize
\definecolor{shadecolor}{rgb}{0.969, 0.969, 0.969}\color{fgcolor}\begin{kframe}
\begin{alltt}
\hlstd{> }\hlkwd{range}\hlstd{(prenoms}\hlopt{$}\hlstd{Annee)}
\end{alltt}
\begin{verbatim}
## [1] 2004 2015
\end{verbatim}
\begin{alltt}
\hlstd{> }\hlkwd{head}\hlstd{(}\hlkwd{names}\hlstd{(tt)[tt}\hlopt{>}\hlnum{12}\hlstd{])}
\end{alltt}
\begin{verbatim}
## [1] "Adama"    "Alix"     "Amelia"  
## [4] "Andrea"   "Ange"     "Angelina"
\end{verbatim}
\end{kframe}
\end{knitrout}



  Oups y'a une petit probl�me dans la base de donn�es.

\begin{knitrout}\footnotesize
\definecolor{shadecolor}{rgb}{0.969, 0.969, 0.969}\color{fgcolor}\begin{kframe}
\begin{alltt}
\hlstd{> }\hlstd{pp} \hlkwb{<-} \hlkwd{names}\hlstd{(tt)[tt}\hlopt{>}\hlnum{12}\hlstd{]}
\hlstd{> }\hlstd{prenoms_prb} \hlkwb{<-} \hlstd{prenoms[prenoms}\hlopt{$}\hlstd{Prenoms} \hlopt \hlstd{pp,]}
\hlstd{> }\hlkwd{head}\hlstd{(prenoms_prb[}\hlkwd{order}\hlstd{(prenoms_prb}\hlopt{$}\hlstd{Prenoms),])}
\end{alltt}
\begin{verbatim}
##      Prenoms Nombre Sexe Annee
## 1268   Adama      5    F  2014
## 2188   Adama     17    M  2008
## 3320   Adama     18    M  2004
## 4817   Adama     15    M  2011
## 5884   Adama      5    M  2012
## 5952   Adama     13    M  2012
\end{verbatim}
\end{kframe}
\end{knitrout}



\begin{knitrout}\footnotesize
\definecolor{shadecolor}{rgb}{0.969, 0.969, 0.969}\color{fgcolor}\begin{kframe}
\begin{alltt}
\hlstd{> }\hlkwd{head}\hlstd{(}\hlkwd{table}\hlstd{(prenoms_prb}\hlopt{$}\hlstd{Prenoms,prenoms_prb}\hlopt{$}\hlstd{Annee))}
\end{alltt}
\begin{verbatim}
##           
##            2004 2005 2006 2007 2008 2009
##   Adama       1    1    1    1    1    1
##   Alix        1    1    1    1    1    1
##   Amelia      1    1    1    2    0    1
##   Andrea      2    2    2    2    2    2
##   Ange        1    1    1    1    1    1
##   Angelina    2    2    2    2    2    1
##           
##            2010 2011 2012 2013 2014 2015
##   Adama       1    1    2    1    2    2
##   Alix        1    2    2    2    2    2
##   Amelia      2    2    2    1    0    1
##   Andrea      2    3    3    2    2    2
##   Ange        1    1    1    2    2    2
##   Angelina    1    2    1    1    1    1
\end{verbatim}
\end{kframe}
\end{knitrout}



\begin{knitrout}\footnotesize
\definecolor{shadecolor}{rgb}{0.969, 0.969, 0.969}\color{fgcolor}\begin{kframe}
\begin{alltt}
\hlstd{> }\hlstd{prenoms_prb[prenoms_prb}\hlopt{$}\hlstd{Prenoms}\hlopt{==}\hlstr{"Andrea"} \hlopt{&} \hlstd{prenoms_prb}\hlopt{$}\hlstd{Annee} \hlopt{==} \hlnum{2012}\hlstd{,]}
\end{alltt}
\begin{verbatim}
##      Prenoms Nombre Sexe Annee
## 6307  Andrea     16    F  2012
## 6308  Andrea     35    F  2012
## 6309  Andrea     11    F  2012
\end{verbatim}
\end{kframe}
\end{knitrout}


